% Created 2012-03-28 Wed 16:08
\documentclass[12pt]{article}
\usepackage[utf8]{inputenc}
\usepackage[T1]{fontenc}
\usepackage{fixltx2e}
\usepackage{graphicx}
\usepackage{longtable}
\usepackage{float}
\usepackage{wrapfig}
\usepackage{soul}
\usepackage{textcomp}
\usepackage{marvosym}
\usepackage[integrals]{wasysym}
\usepackage{latexsym}
\usepackage{amssymb}
\usepackage{hyperref}
\usepackage{enumitem}
\tolerance=1000
\usepackage{amsmath}
\usepackage[T1]{fontenc}
\usepackage{mathpazo}
\usepackage[scaled]{helvet}
\usepackage{courier}
\usepackage{natbib}
\setlength{\parindent}{0.0in}
\setlength{\parskip}{0.0in}
\usepackage{setspace}
\onehalfspacing{}
\usepackage[raggedright,bf,sf]{titlesec}
\usepackage{fullpage}
\usepackage{xcolor}
\usepackage{listings}
\renewcommand{\maketitle}{}
\usepackage{float}

\floatstyle{boxed}
\restylefloat{figure}


\definecolor{lightgray}{gray}{0.95}
\lstnewenvironment{code}[1][]%
  {\minipage{\linewidth}
\lstset{language=,
          keywordstyle=\bfseries,
          captionpos=b,
          backgroundcolor=\color{lightgray},
          frame=shadowbox,
          rulesepcolor=\color{gray},
          basicstyle=\ttfamily\fontsize{10}{10}\selectfont,
          aboveskip=20pt,
          literate={æ}{{\ae}}1
                   {ø}{{\o}}1
                   {å}{{\aa}}1
                   {Æ}{{\AE}}1
                   {Ø}{{\O}}1
                   {Å}{{\AA}}1,
          #1}}
  {\endminipage}


\begin{document}

\maketitle

\newcommand{\blankpage}{\newpage{}\thispagestyle{empty}\mbox{}\newpage{}}
\newcommand{\HRule}{\rule{\linewidth}{0.5mm}}

\begin{titlepage}
\begin{center}
\includegraphics[width=8cm]{pic/uib-emblem-svart} \\[0.5cm]
\paragraph*{}

\textsc{\Large INFO212}\\[0.5cm]
\Large Induvidual Assignmnet\\[0.4cm]
\HRule{}\\[0.4cm]
{\huge \bfseries Open-Source Development}\\[0.5cm]
\HRule{}\\[1.0cm]

\emph{Candidate number:\\
      176}\\

      \paragraph*{}
      \end{center}
      \vfill
      \begin{center}
      {\large \today}
      \end{center}
      \end{titlepage}

      \setcounter{tocdepth}{3}
      \tableofcontents

      \clearpage

      \setlength{\parskip}{0.2in}
      \pagenumbering{arabic}



\section{Introduction}
Open-Source software is among the most important software in use today. It
powers close to 86\% of all mobile phones\cite{idc}, and its estimated that
about 75\% of the worlds top websites are run by open-source
software\cite{pingdom}. In this essay we will take a look on how open-source code
is written, the tools used and the strategy, along with a few examples of them
in use. We will also touch on some of the debate around open-source software
versus free software.


\section{Open-Source}
Open-source has come a long way inn general. This essay is written on largely
free and open-software, with a few exceptions. The computer runs Linux, one of
the largest open-source projects inn the world. The essay is written inn Vim,
descendants from one of the oldest editors in the world.  It's written with
latex, which uses a open-source toolchain and compiles down too the open PDF
standard. Some of these tools are older then me, some are barley younger. With
the rise of social coding platforms, we have seen an increase inn code written
openly on the web.

However, redistribution of code isn't a new thing. It's been prominent inn the
academia for many years prior to the 1980. As well as the rise of the homebrew
computer club inn Palo Alto during the 1970s. Arguably, the world wide web as we
know it today is built on the very idea of a collaborated effort to share ideas
and code. CERN gave the original web server and web browser away for
free\cite{cern}. They wrote their own license for this that later became one of
the most used licenses today, the MIT license. We will take a look at it later.

Richard Stallman established the Free Software Foundation and the GNU Project
inn 1983.  The GNU project had a goal to write a complete open and free
operating system, as most operating systems during this era was proprietary and
not open-source\cite{fsf}. A popular example is your printer does not work, you could fix
it if the source was open.  But with a proprietary system this was not possible.
Leaving you with a broken system with no other way to fix it but to contact the
manufacturer.  Stallman was out to change this. They created many of the tools
commonly used for software development today.  The concept of open-source was
still in its infancy during the 1990s, it was not before Linux arrived in 1991
that the GNU project was able to fulfill their goal of creating a complete open
and free operating system.


\subsection{Free versus Open-Source Software}
There are two large camps in the world of Free and Open-Source software. On one
side you have Richard Stallman with his Free Software Movement, and on the other
side Eric S. Raymond with the Open-Source initiative. The Free Software Movement
redefines the term "free" to the word of "libre"\cite{fsf}, you can modify the code, but
you have to redistribute the code and make it open. They reject the notion that
the use of computers should prevent people from co-operating. This leads to the
rejection of proprietary software that Stallman very heavily argues against.

The Eric S. Raymond however defined the term Open-Source. Following his essay
"The Cathedral and the Bazaar" Netscape released their source code of the
netscape browser. The advocacy and the philosophies of the Free Software
Foundation didn't appeal to companies. So they wanted a better term that could
incorporate the ideas, yet be friendly to companies. The Open Source Initiative
was established. They published guidelines inspired by the Free Software
foundation, yet more open and less hostile\cite{osi}.

It is important to realize that the terms "open-source" and "free software" have
a very rigid definition. These definitions are heavily represented by the
licenses they push forward. GPL for the Free Software Foundation, and the
MIT/BSD/OSI approved licenses from the Open Source Initiate.


\subsubsection{Licenses}
Licenses are important in the world of open-source. Normally the writer retains
the copyright of the code written, making any open code written without a
license on shaky ground as you are at the mercy of the writer, and not the
license.

There are two types of licenses. Free, or libre, licenses like the GPL,
and the permissive licenses like BSD and MIT. And depending on what license you
use, you might not be able to use the code for your software.

Stallman invented the widely used GPL license. GPL allows the end use to modify
and redistribute the code. It is also a copyleft license which means
derivative work has to be redistributed with the same license. This goes both
ways. This makes it harder to developers to figure out what licenses are
compatible with each other. Microsoft for instance has been very harsh critics
of GPL, as any libraries written with GPL licensed code has to be redistributed
as such.

On the other side of the specter, we have the BSD and MIT licenses. Both
licenses put a very limited set of restrictions on reuse, and in turns makes it
compatible with other licenses such as GPL. MIT and BSD allows anyone to do
anything with the provided software, but the originally writer can not be hold
liable for any damage cause by the software. The simplicity of the licenses
makes it the most widely used licenses today.

The interesting part of this is that for you to call your project "Open-source",
you would need to use a OSI (Open Source Initiative) approved
license\cite{osi-license-list}. The Free Software Foundation and the Open Source
Initiate both publish a list of licenses they approved\cite{fsf-license-list}.
Both the GPL and the MIT license are both approved by the FSF and OSI alike. So
anyone could use either licenses and mix them in their project, but you would be
forced by the GPL to redistribute the code. There is also a concept about
dual-licenses, where as you can licenses one part of your code as MIT, and
another as GPL. But we wont dive to much into this in this essay.


The discussion about using a permissive license of a libre license is today a
very big and political discussion. Some people believe a license like GPL
protects the freedom of the user, because of redistribution. While other choose
something like the BSD or the MIT because they equal "true" freedom. This is
largely a ongoing argument in the deep ends of the Internet.

Today MIT is the most widely used license for open-source work. However, several
large projects use the GPL as its more reasonable when you want contributions in
return. One example of this is Linux.


\subsection{The Cathedral and the Bazaar}
One of the largest influences on the Open-Source community inn the early 2000
was the essay ``The Cathedral and The Bazaar'', referred to as ``CatB'' written
by Eric S. Raymond\cite{catb}.  The essay looks at the development of the Linux kernel,
which we will take a look at later in this essay, and his own experiences
developing fetchmail. CatB was a large influence, it helped Netscape open-source
their code for the Netscape Webbrowser, which formed Firefox and started the
Mozilla project. It gave rise to several prominent projects.

The essay proposes that ``Given enough eyeballs, all bugs are shallow'', which
means that given enough developers and testers, all bugs are easy to fix and
correct. This could be in the form of code reviews where several developers look
at the written code, and tests it. Thus its easier to fix and find flaws inn the
code early on. This has later been formulated as Linus's Law.

The essay explains the difference between two models of developing free and open
software, The Cathedral and The Bazaar model of development.

\subsubsection{The Cathedral}
The Cathedral model is where the source code is available between every release
of the software. The system is thus only shared between a number of developers
in private. This means you can't look at code in between releases. You could
look at a log of all the changes and not really know what code went where. This
style has largely fallen away with the rise of version control systems, but is
still used in some software, notably TrueCrypt, a now deprecated disk encryption
software, and Putty, a widely used software to managed servers with SSH.


\subsubsection{The Bazaar}
The Bazaar models is how most of the open-source work today is done today. The
idea is to make development transparent and open. You should be able to read the
changes, see the changes and follow the development of the project. With the
spawn of social programming websites like bitbucket, gitlab and github this is
easier then every before.  This ranges from the Linux kernel, several prominent
programming languages such as Golang, Rust, Clojure, Python and Ruby. Among
compilers for C and C++. It's overall a very successful model of development
and with the rise of social programming platform it become easier to start
contributing.

This model allows for what Eric S. Raymond dubbed Linus's Law. As the code is
always available, at any state during the development. It is easier to debug,
test and get the needed reviews for the code.


\section{Git} 
Version control software, VCS for short. Is a program that keeps track of
changes inn a software project. It keeps track of added and deleted lines of
code in a folder, and enables you to commit these changes to a repository. These
repositories then lets you have multiple parallel sets of changes into different
branches. This enables developers to keep track of features, changes and version
their software to a greater extent. This is crucial in todays software
development.

Git is a distributed version control system created by Linus Torvalds\cite{git}.
It was mainly created to replace the proprietary solution used by the Linux
developers.  The goal of the project is to create distributed version control
system. By being distributed developers could stay offline for an extended
period of time, working on their own local copy of code, and merge it back
together when needed, instead of having to stay in-sync with the current status
of the code. This is a interesting feature compared to SVN, which is another
often used VCS. SVN only allows commits to the trunk if you can reach the source
repository.  This prevents developers from working on the project away from
work, if the repository is behind a firewall. With the distributed nature of
Git, this is not a problem.


\section{Social Coding Platforms}
Social coding platform has a been a game changer for the free and open source
software world. More software is getting released now then ever, and its largely
because of websites and services supporting these activities. One of the largest
of these sites is Github. Github allows you to create free open repositories for
your code. They also introduce an option to fork other project, simply to clone
and host a copy on your own profile, and send back pull-requests. These
pull-requests lets you contribute with changes to a project you don't have
direct access to. This way you can propose changes, figure out bugs or errors
early on.

This adheres to the Bazaar model discussed earlier. Everyone is able to review
the code and changes. The changelogs for the project could reference these
commits, and pull-requests which as well doubles as a discussion platform for
the proposed changes. A lot of large projects use this for their discussions.
Docker is a software allowing developers to create lightweight containers. Rust
is a up and commit low-level programming langauge, and Golang, a new
programming language from google. All these projects use Pull-Requests as a
discussion platform.

Github started out inn 2007 as a part-time project by Tom Preston-Werner. Today
it hosts around 49 million projects, for its estimated 14 million
users\cite{github-about}. It has also spawned a pawed way for a lot of the
projects and conferences we see today.  There has also been created alternatives
for Github with different models.

Github allows users to create unlimited code repositories. To create private
ones you have to pay. In contrast, bitbucket allows anyone to great private
repositories, but limits the amount of contributors that share any repository.
This enables students with projects to utilize bitbucket better then github, as
you want your projects to not be visible.

There are also open-source github clones, like gitlab. They allow you to
self-host your own gitlab instance. This is great for a company that can't use
any external services for their code. There are also several smaller projects
that enables you to just display information from git without any extra features
like bug trackers.


\subsection{Development Methods}
Git itself does not enforce any way of how development should be done. A lot of
people working alone on projects commits straight to the  master branch. While
some projects utilize other models on how code should be commited.

A few popular examples are the git-flow model\cite{git-flow}, and feature-branches.
Feature-branches is essentially a way of saying that one branch should contain
one feature, or one new addition to the project. This enables you to manage each
merge as a feature, and easily revert changes that should not work out. This
also enables you to expand into more complex workflows, like git flow.

Git flow is a structure where you have 2 long lived branches. One master, used
as a production branch. All code going into this branch should be vetted, and
deployable.

\subsection{Tools}
There is a large variety of tools available to compliment the development of
open-source software. This could be in the form of supporting testing and
integrations. Keeping track of bugs, and issues. There are also numerous systems
to help developers and projects to deploy into production systems. We will take
a look at different tools that support developers of open-source projects and
companies in this section.


\subsection{Bugtracking}
Bugtracking is an essential part of a open-source project. It enables you to
label, track, and assign bugs, issues or planned features of a project. Linux
uses Bugzilla with tight integration to the mailing list to track bugs. This
enables the developers to keep their main attention to the mailing list. Github
includes a bugtracker, but some projects feel it's bad to centralize everything
and choose other providers. This helps a lot if it would ever happen that github
goes down. The Tor Project creates privacy enhancing tools for the Internet and
relies on the open-source trac bugtracker for their issues and bugs.


\subsection{Testing \& Build tools}
Testing and building is essential for any projects, not only open-source. It
enables the project to verify and test the code they are writing.

Buildbots is a open-source build automation software\cite{buildbot}. It enables the project
creator to keep build created along with the code commits to the repository.
This is heavily used by projects like chromium, googles open-source web browser
which chrome is built on. This lets chromium to build and offer experimental
builds straight from their git repository, without having the user to manually
clone and build the webbrowser them self, which can be a complicated task.

Travis is a proprietary product that integrates heavily into github\cite{travis}. It enables
both public and private repositories to create multiple testing suites for their
project. This is great as feedback can be directed into the pull-request on the
github website, instead of being a separate process. This makes it easier for
the individual contributor to receive feedback and check the errors.

Jenkins is a widely used open-source tool\cite{jenkins}. It allows you to setup testing and
deployment for your systems. Since it can be self-hosted it is also heavily used
internally by companies for testing and deployment. It is written in java and
has a great open-source ecosystem for integration between different systems. You
could setup git integration between github and bitbucket. Do the testing using
docker container and scale across a cloud. And when done, you could deploy your
product to the cloud provider of choice; google, amazone or azure.

There is also an own category of testing tools for code. These are called
linters and are used to catch coding mistakes, and code not conforming to a
given coding standard. These can be used along with tools like jenkins to make
sure the code submitted is within whats accepted of the community.


\subsection{Containers and virtual environments}
Inn recent years technology like containers and virtual environment has emerged
from open-source communities. Technologies like Vagrant and Docker enables
developers to not care about the underlying operating system they develop on.
But can initiate thin layers to emulate different operating systems. This
enables rapid prototyping and easier testing as the environment does not have to
be created, but can be emulated with configuration files.

Several tools like Travis and Jenkis has started to rely on this technology as
it scales easier between machines and enables you to create small clouds. This
can make testing for complex software that needs multiple concurrent test more
efficient.

\subsection{Package management}
Package management is important in the development of
software\cite{packagemanager}. Packages are one application, its dependency
list, its source files and usually its compiled form. This enables you to keep
track of the installed software, libraries and needed dependencies on your
computer. They are usually stored in a centralized repository. This is crucial
in today open-source world as installing this by hand is hard, and often
complicated. A single dependency for your project could contain 3 dependencies,
who might have 12 other dependencies together. A package manager simplifies this
process. The only thing required by the author is to make there is a updated
list of dependencies for the project.

You can divide between two package managers. System package managers and
application-level package managers. System package managers are often the
package manager used in your operating system. This depends on the Linux
distribution, or if its macOS or windows. Neither macOS nor Windows was designed
to be used with a package manager, but solution has been developed. Respectively
homebrew and chocolatey.

Application-level package manager support packages for their own application.
This is commonly used to manage dependencies for popular programming languages
like, Python, PHP, NodeJS, Ruby and Perl. These are usually called by the system
package managed so they don't conflict with the different operating system.

\section{Communication}
Communication is important in the open-source world. There has to be a place for
the developers to discuss features, the road ahead and how things should
progress. Even with the rise of social coding platform, two things has largely
stayed the same; Mailinglists and IRC.

Mailinglists is used a lot inn projects. This is where the announcements go, and
general discussion of features. Patches can be sent here, help and guidance is
also often here.

IRC is one of the oldest chatting services in the world. It was created inn 1988
by a finnish student. It has been one of the most important chatting platform
through the years. However, it has declined inn more recent years. But
developers and IT people still enter and actively discuss on these
networks\cite{irc}.  Larger projects use these channels to casual chitchat,
discussions and meetings.  The Tor project heavily use these channels to plan
development.

Conferences is also an important part of the open-source ecosystem. Since large
parts of the developers do not work together physically, conferences is where
they meet and discuss. One of the largest is called FOSDEM\cite{fosdem}, and currently
hosts 8000 people from the open-source community inn a free today event. Here
you can attend workshops, talk with contributors from other projects, and join
sprints if you want to try and get your hands on some new projects.

\section{Linux}
Linux is the largest open-source project ever created. The project was started
in 1991 by Linus Torvalds to create an open clone of the MINIX kernel, only
available for educational use\cite{history-of-linux}. As Linus is only an
operating system, it is supported by the GNU project, providing the user tools
to interact with the operating system. This has created some controversy around
the naming of the operating system as Stallman with the Free Software Foundation
claims it should be named GNU/Linux as linux alone is not in reality an
operating system.

Linux has so far more then 3000 developers, and over 400 companies making it one
of the most successful open-source projects ever. Linux is behind the largest
market share of mobile phones, and web servers, driving the world wide
web.


\subsection{Development Method}
Linux is developed by using a distributed version control system called git. The
community discussion is centered around mailing lists and IRC channels, a
distributed messaging protocol. Linux has a lot of maintainers and developers
working on the project, 13\% are volunteers, and rest is from companies like
Google, IBM and Redhat. The release cycle lasts about 2 months, where the
patches has a review period of 2 weeks. All patches are generated and sent to
the mailing, and reviewed there\cite{linux-maintainer}.

Linux does not use a social coding platform for its code. There is a mirror
available on github, but they do not accept changes there. They host their own
git repository on their own website.

Because of the large codebase, linux has a lot of maintainers spread out over
different sub-systems of the kernel. One person could manage the file system
drivers, while another would maintain the wireless drivers. During a normal
2 weeks period developers can receive as much as 490 patches, and the 3.8 release
of linux had 6-7 changes per hour applied. The most recent release of linux,
version 4.9, had the most changes ever. 16,216 in the 4.9 release cycle. Linux
release a new version every month, this means that there is roughly 270 changes
a day to the linux kernel at its current state\cite{linux-49}.

\subsection{Distributions}
Linux and the GNU project itself alone is a boring thing. It would largely only
give you a shell on a black screen with a cursor to type inn commands. It's the
distributions of Linux that really makes it. A distribution is essentially
Linux+GNU+More. They together create a graphical environment along with a
package manager. Depending on the distribution used, different settings and
default packages could be installed. Some distribution only give you the bare
minimum needed for you to create your own tailored system, Arch Linux and Gentoo
are good examples here. While others like Ubuntu, Debian or Linux Mint create a
very complete feeling of an operating system.

They can provide a installation CD, guiding you through the process of
installing the system, with a default set of configurations. This makes linux
largely accessible for a wider audience. The minimal distributions also suits
more experienced developers and power-user better.

\subsubsection{Debian}
Debian is one oldest distributions of linux. It was created by 1993 by Ian
Murdoc, and the first stable release appeared in 1996. Today its one of the most
widely used distribution of Linux.\cite{debian}

Debian largely inspired the works of the Open Source Initiative by Eric S.
Raymond. They defined "Debian Free Software Guidelines" which is part of their
social contract. This determines if a license is an Open Source license, and can
be used inside the Debian project, The Open Source Initiative adopted these
guidelines when they later defined the open-source license approvals.

Debian is largely built by people inn their free time. It is estimated that the
work put into the project is worth around 8 billion US Dollars.

It powers home computers, server providers and even the international space
station. NASA changed from Windows too Debian for their space station inn 2013.
Largely because of stability and ease of management and updating\cite{nasa}.

\section{VLC}
VLC is a free and open-source video player. It supports a large number of
codecs and medias to play from. It's multi-platform and supports Linux, Windows,
macOS, Android and iOS among others. It is widely used for being free and
supporting the most used video formats today. 

It started out as way for the campus students to stream from satellites over a
campus network, but today is a more generalized media playback service. People
use the software today for media playback. It has been downloaded more then 2
billion times


\subsection{Development}
VLC uses a workflow very similar to the workflow of Linux. They are based
around mailing lists, use an external bugtracker, in this case trac. And utilize
git for its version control. The code is hosted on an external git server, and
they do not relay on external services such as github.  There is also a wiki
website to collect developer information, and misc information about the
project. It is released under the GPL license.


\section{Tor}
Tor is an anonymity network created and maintained by The Tor Project.  The
network consists of nodes hosted by anyone, worldwide and through cryptography
keeps the origin of the connection to the destination a secret. It was
originally developed to be used by the US armed forces, but since been
open-sourced and improved upon. It is today widely used by journalists and
outspoken activities to hide their identity\cite{about-tor}.

Tor also allows people to host "Hidden Services", which is webpages only
accessible from the Tor network. This enables the services to hide their IP, and
the services to never know who accessed the websites.

Tor has gained a lot of controversy as people have been using the system for
distribution of illegal items. Silk Road, an infamous drug dealing website was
funded on the Tor network and gave the concept of "dark nett" to the public
media.

\subsection{Development}
The development of tor is slightly different from what we have seen. They do
host their own git server, with a large collection of projects that belong to
contributors. The main discussion happens on mailing lists, but patches and
proposed code changes are never submitted there.

Instead they utilize the trac bugtracking software for the submission of
changes. This differs from Linux and VLC which we went through earlier. This
choice makes it easier for people not following the wast mailinglists of the
project, and it easier for new people to contribute. A lot of the discussion and
the community effort is done over IRC, they have several channels dedicated for
Tor users to talk inn. They also provide own developer channels for new people
that want to contribute to one of the projects under tor.


\section{Chromium}
Chromium is a open-source webbrowser released by google. It is used as the
development version to googles stable webbrowser, chrome. It is considered to be
among the most secure webbrowser today, and there is a yearly competition to
find bugs, or zero day exploits inn the software. The bounty is currently at
63,000 US Dollars\cite{pwn2own}.

\subsection{Development}
Chromium is written on a rolling release basis. That means every commit into the
project is built and handed out to users. This is know as Continuous
Integration, along with using Buildbot to build and create available version,
they also use a Continuous Build system. This enables the chromium to adapt
quickly whenever there is a bug, or much needed feature available for the user.

Large parts of the chromium discussion is on their IRC channel, along with
mailinglists. They also provide a self-made codereview tool for submitted
patches from the community.


\section{Summary}
In this essay we have taken a look at important aspects of the open-source
community, and the development of open-source software and the history behind
it. Free and Open Source Software is crucial and important in todays technology
driven world. The projects fueled by the work of volunteers and corporations
alike is important for the growth of the internet.

\clearpage
\bibliography{oppgave}
\bibliographystyle{unsrt}
\end{document}
